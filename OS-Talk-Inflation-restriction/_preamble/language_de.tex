%%% Language preamble for german

% Language itself
\usepackage[ngerman]{babel}
% Font encoding to represent umlauts correct in PDFs documents instead of combining them from other
% characters like "u instead of ü
\usepackage[T1]{fontenc}
% Encoding of the source code.
\usepackage[utf8]{inputenc}


% Language specific strings
%   Names of theorem environments

\addto\captionsngerman{
	% \see command from makeidx
	\@ifpackageloaded{makeidx}{
		\renewcommand{\seename}{siehe}
	}{}
	
	\renewcommand{\captionstringtheorem}{Satz}
	\renewcommand{\captionstringlemma}{Lemma}
	\renewcommand{\captionstringcorollary}{Korollar}
	\renewcommand{\captionstringlemmadef}{Lemma und Definition}
	\renewcommand{\captionstringtheoremdef}{Satz und Definition}
	\renewcommand{\captionstringdefinition}{Definition}
	\renewcommand{\captionstringproposition}{Proposition}
	\renewcommand{\captionstringexample}{Beispiel}
	\renewcommand{\captionstringconjecture}{Vermutung}
	\renewcommand{\captionstringconvention}{Vereinbaring}
	\renewcommand{\captionstringremark}{Bemerkung}
	
}
\@ifpackageloaded{biblatex}{
\DefineBibliographyStrings{german}{
	bibliography = {Bibliographie},
	references = {Referenzen},
}
}{}
