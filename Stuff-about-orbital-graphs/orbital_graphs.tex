% !TeX root = orbital_graphs.tex
% !TeX spellcheck = en_GB
\documentclass[fontsize=11pt,fleqn,a4paper]{scrartcl}
%%% Language preamble for english

% Language itself
\usepackage[english]{babel}
% Font encoding to represent umlauts correct in PDFs documents instead of combining them from other
% characters like "u instead of ü
\usepackage[T1]{fontenc}
% Encoding of the source code.
\usepackage[utf8]{inputenc}


% Language specific strings
%   Names of theorem environments
%\newcommand{\captionstringtheorem}{Theorem}
%\newcommand{\captionstringlemma}{Lemma}
%\newcommand{\captionstringcorollary}{Corollary}
%\newcommand{\captionstringlemmadef}{Lemma and definition}
%\newcommand{\captionstringtheoremdef}{Theorem and definition}
%\newcommand{\captionstringdefinition}{Definition}
%\newcommand{\captionstringproposition}{Proposition}
%\newcommand{\captionstringexample}{Example}
%\newcommand{\captionstringconjecture}{Conjecture}
%\newcommand{\captionstringconvention}{Convention}
%\newcommand{\captionstringremark}{Remark}

%%% General math packages

% AMS
\usepackage{amsmath,  
            amssymb,  % Symbols
            amsthm,   % provides theorem environments
            amsfonts  % fonts like \mathbb and \mathfrak
            }

% useful things for math typesetting like \smash, \psmallmatrix, ...
\usepackage{mathtools}

% For \Set and \set. Automatic resizing of the curly braces and the middle vertical line
\usepackage{braket}   

% For even more extensible arrows
\usepackage{extpfeil} 

% Blockmatrices
%\usepackage{multirow}

% and brakets around matrix rows
%\usepackage{bigdelim}


%%% Own symbols and operators
\newcommand{\IN}{\mathbb{N}}
\newcommand{\IZ}{\mathbb{Z}}
\newcommand{\IQ}{\mathbb{Q}}
\newcommand{\IR}{\mathbb{R}}
\newcommand{\IC}{\mathbb{C}}
\newcommand{\IK}{\mathbb{K}}

\DeclarePairedDelimiter{\abs}{\lvert}{\rvert}
\DeclarePairedDelimiter{\norm}{\lVert}{\rVert}
\DeclarePairedDelimiter{\ceil}{\lceil}{\rceil}
\DeclarePairedDelimiter{\floor}{\lfloor}{\rfloor}

\newcommand{\isomorphic}{\cong}
\newcommand{\homotopic}[1][]{\overset{#1}{\simeq}}

\newcommand{\colim}{\mathop{\rm colim}\limits}


\renewcommand{\Im}{\operatorname{\mathfrak{Im}}}
\renewcommand{\Re}{\operatorname{\mathfrak{Re}}}

\DeclareMathOperator{\id}{id}
\DeclareMathOperator{\Hom}{Hom}
\DeclareMathOperator{\End}{End}
\DeclareMathOperator{\Irr}{Irr}

\DeclareMathOperator{\Ind}{Ind}
\DeclareMathOperator{\Res}{Res}

\DeclareMathOperator{\tr}{tr}
\DeclareMathOperator{\sgn}{sgn}
\DeclareMathOperator{\diag}{diag}
\DeclareMathOperator{\ord}{ord}

\DeclareMathOperator{\CharFld}{char}
\DeclareMathOperator{\QuotFld}{Quot}

\DeclareMathOperator{\im}{im}
\DeclareMathOperator{\rad}{rad}

%%% Theorem environments

% Define name strings
\newcommand{\captionstringtheorem}{Theorem}
\newcommand{\captionstringlemma}{Lemma}
\newcommand{\captionstringcorollary}{Corollary}
\newcommand{\captionstringlemmadef}{Lemma and definition}
\newcommand{\captionstringtheoremdef}{Theorem and definition}
\newcommand{\captionstringdefinition}{Definition}
\newcommand{\captionstringproposition}{Proposition}
\newcommand{\captionstringexample}{Example}
\newcommand{\captionstringconjecture}{Conjecture}
\newcommand{\captionstringconvention}{Convention}
\newcommand{\captionstringremark}{Remark}


% Definition of two similar styles
\newtheoremstyle{dotless} % Name
			{\bigskipamount}    % Space above
			{\bigskipamount}    % Space below
			{\nopagebreak}      % Body font, also suppress pagebreak between "Theorem 3.14:" and text
			{}                  % Indent amount
			{\bfseries}         % Theorem head font
			{:}                 % Punctuation after theorem head
			{\newline}          % Space after theorem head
			{}                  % Theorem head spec (can be left empty, meaning 'normal')
\newtheoremstyle{dotless2} % Name
			{\bigskipamount}    % Space above
			{0.0em}             % Space below
			{}                  % Body font
			{}                  % Indent amount
			{\bfseries}         % Theorem head font
			{:}                 % Punctuation after theorem head
			{0.5em}             % Space after theorem head
			{}                  % Theorem head spec (can be left empty, meaning 'normal')
% "3.14 Theorem" instead of "Theorem 3.14"
\swapnumbers

\newcounter{theoremnumber}
\numberwithin{theoremnumber}{section}

\theoremstyle{dotless}
\newtheorem{theorem}[theoremnumber]{\captionstringtheorem}
\newtheorem{theoremdef}[theoremnumber]{\captionstringtheoremdef}
\newtheorem{lemma}[theoremnumber]{\captionstringlemma}
\newtheorem{lemmadef}[theoremnumber]{\captionstringlemmadef}
\newtheorem{corollary}[theoremnumber]{\captionstringcorollary}
\newtheorem{proposition}[theoremnumber]{\captionstringproposition}
\newtheorem{definition}[theoremnumber]{\captionstringdefinition}
\newtheorem{example}[theoremnumber]{\captionstringexample}
\newtheorem{conjecture}[theoremnumber]{\captionstringconjecture}

\newtheorem*{convention}{\captionstringconvention}

\theoremstyle{dotless2}
\newtheorem{remark}[theoremnumber]{}


%%% Math style file

% Sets the strictness of handling page breaks inside align-Umgebungen
% Level [1] breaks will be avoided if possible,
% Levels [2], [3], [4] are increasingly relaxed.
\allowdisplaybreaks[1]


%%% Everything tikz

\usepackage{tikz}
\usetikzlibrary{arrows}
\usetikzlibrary{cd}

% Predefine styles
\tikzset
{
	desc/.style=
	{
		fill=white,inner sep=2pt,font=\scriptsize
	}
}


%%% Styles for algorithms

\usepackage{listings}
\lstset{%
	basicstyle = \ttfamily\small,
	tabsize = 3
}
% Use theorem environment for algorithm descriptions
\newtheorem{algorithm}{Algorithmus}
% Change numbering of algorithms to include chapter
\renewcommand{\thealgorithm}{\Alph{chapter}.\Roman{algorithm}}
%%%% Tables and figures
\numberwithin{table}{section}
\numberwithin{figure}{section}



%%% Text styles

% For the interrobang
\usepackage{textcomp}

% Additional underlining options, especially dotted underlining, line breaks in underlined text etc.
\usepackage[normalem]{ulem} % option not to change look of \emph{}
\newcommand*{\udot}{\dotuline}


% Skip lengths
\setlength{\parindent}{0em}
\setlength{\parskip}{0em}


% Pimp enumerate and itemize environments. More counter options and resuming numbering
\usepackage{enumitem}

% Style of enumerations and itemizations
\renewcommand{\labelenumi}{\alph{enumi}.)}  % Counter enumi wird immer als a.) b.) c.) dargestellt.
\renewcommand{\labelenumii}{\roman{enumii}.)}  % Counter enumii wird immer als i.) ii.) iii.) dargestellt.


\usepackage{csquotes}

\usepackage{multicol}

% landscape pages
\usepackage{pdflscape}

% For colorful notes in the margin
% Load after tikz!
\usepackage[backgroundcolor=red!20,textsize=footnotesize]{todonotes}
%%%% Bibliography with BibTeX only

\usepackage[numbers]{natbib}
\bibliographystyle{plain}


%%%% Bibliography with biblatex + bibtex

\usepackage[
	style=alphabetic,
	sorting=nty,
	url=false,
	natbib=true,
	backend=biber,  % Biber backend
	defernumbers=true
]{biblatex}
% The .bib-file(s) for this document
\addbibresource{wgraph.bib}


%%%% Bibliography with biblatex + bibtex

\usepackage[
	style=alphabetic,
	sorting=nty,
	url=false,
	natbib=true,
	backend=biber,  % Biber backend
	defernumbers=true
]{biblatex}
% The .bib-file(s) for this document
\addbibresource{wgraph.bib}



% !! Hyperref before imakeidx !!
%%% PDF stuff
%%%
%%% Include hyperref before imakeidx !!

\usepackage[
	pdfpagelabels=true,
	plainpages=false
	]{hyperref}
	
\hypersetup{
colorlinks=true,
linkcolor=red,
urlcolor=blue,
citecolor=blue,
linktocpage=true, % Page numbers will be the links in t.o.c instead of the headings themselves
pdfpagelayout={OneColumn},
pdfstartview= % empty to cause the viewer to use its preferred behaviour instead of dictating a behaviour at opening of the document
}


%\input{_preamble/indicies.tex}


\makeatletter
\hypersetup{
pdfinfo=
	{  
		Title={\@title},
		Author={\@author},
		Keywords={Permutation group, orbital graph, representation theory},
		Subject={group theory}
	}
}
\makeatother

\title{Some stuff about orbital graphs}

\begin{document}

\maketitle

\begin{definition}
Let $G\leq Sym(\Omega)$ be a permutation group. Then an orbit of $G$ on $\Omega\times\Omega$ is called an \udot{orbital} of the action $G \curvearrowright \Omega$.

If $\Gamma\subseteq\Omega\times\Omega$ is an orbital, the directed graph with vertex set $\Omega$ and edge set $\Gamma$ is called an \udot{orbital graph}.
\end{definition}

\begin{definition}
Let $G\curvearrowright\Omega$ be a transitive action. A $G_\omega$-orbit of this action is called a \udot{suborbit}. The sizes of the suborbits are called the \udot{subdegree}s of $G$.
\end{definition}

\section{Orbital graph, suborbits and double cosets}

\begin{theorem}
Let $G \curvearrowright \Omega$ be transitive, $\omega\in\Omega$ any element and $H:=G_\omega$ its stabiliser.

There are inclusion-preserving bijections between the following sets
\begin{enumerate}
\item $G$-invariant subsets $\Gamma\subseteq\Omega\times\Omega$.
\item $H$-invariant subsets $\Delta\subseteq\Omega$
\item Subsets $D\subseteq G$ invariant under left- and right-multiplication by $H$.
\end{enumerate}
given in the following dictionary
\[\begin{array}{c|c|c}
\Gamma\subseteq\Omega\times\Omega & \Delta\subseteq\Omega & D\subseteq G \\
\hline
\Gamma & \Gamma(\omega) := \set{\alpha | (\alpha,\omega)\in\Gamma} & \set{ y\in G | (\omega,{^y \omega})\in\Gamma} \\
\set{({^g \alpha},{^g \omega}) | \alpha\in\Delta,g\in G} & \Delta & \set{y\in G | {^y \omega}\in\Delta} \\
\set{({^{g_0}\omega}, {^{g_1}\omega}) | Hg_0^{-1} g_1 H \subseteq D} & {^D\omega} & D \\
\end{array}\]
In particular the minimal non-empty elements of these posets, namely the orbitals, the suborbitals and the $H$-$H$-double cosets respectively, are mapped bijectively onto each other.

Moreover, these bijections translate the following properties:
\[\begin{array}{c|c|c}
\Gamma\subseteq\Omega\times\Omega & \Delta\subseteq\Omega & D\subseteq G \\
\hline
\set{(\alpha,\alpha) | \alpha\in\Omega} & \set{\omega} & H \\
\Gamma^{op} & \Delta^\ast & D^{-1} \\
\abs{\Gamma}/\abs{\Omega} & \abs{\Delta} & \abs{D}/\abs{H} \\
\Gamma \circ \Gamma' & \Delta \circ_\omega \Delta' & DD'
\end{array}\]
where
\[\Delta^\ast :=\set{{^{g^{-1}}\omega} | {^g \omega}\in\Delta}\]
\[\Delta \circ_\omega \Delta' := \set{\alpha | \exists g\in G, \beta\in\Delta': {^g \alpha}\in\Delta \wedge {^g\beta}=\omega}\]
\end{theorem}
\begin{proof}
\end{proof}

\begin{corollary}
Let $G \curvearrowright \Omega$ be transitive.

\begin{enumerate}
\item The connected components of the orbital graph $(\Omega,\Gamma)$ with associated double coset $HyH$ are exactly the $U$-orbits, where $U:=\langle H,y\rangle$.
\item $(\Omega,\Gamma)$ is connected iff $\langle H,y\rangle = G$.
\item $G$ acts primitively iff all non-trivial orbital graphs are connected.
\end{enumerate}
\end{corollary}
\begin{proof}
Identify $\Omega$ with $G/H$ and $\Gamma$ with $\Gamma_y$ as above. Set $U:=\langle H,y\rangle$. Note that $U=H\cup HyH\cup HyHyH\cup\ldots$ because the order of $y$ is finite.

\medbreak
a. Now $xH$ and $x'H$ are connected iff there exists a sequence $x=x_0,x_1,\ldots,x_k=x'$ such that $(x_{i-1}H,x_i H)\in\Gamma_y$, i.e. $x_{i-1}^{-1} x_i \in HyH$.

In particular: If $xH$ and $x'H$ are connected, then $x_{i-1}U = x_i U$ for all $i\in\set{1,\ldots,k}$. Therefore $xU=x_0U = x_k U = x'U$.

Conversely: If $xU=x'U$, then there exists an element $h_0 y h_1 \cdots y h_k \in U$ with $h_i\in H$ such that $x' = x(h_0 y h_1 \cdots y h_k)$. Now we can define $x_i := x \cdot (h_0 y h_1 \cdots y h_i)$ for $i\in\set{0,\ldots,k}$ and have found a sequence connecting $xH=x_0H$ and $x'H$ in the orbital graph.

\medbreak
b. follows directly from a.

\medbreak
c. follows directly from the fact that $G$ acts primitively on $\Omega$ iff $H$ is a maximal subgroup.
\end{proof}

\begin{remark}
This lemma allows for easy identification of at least one block system for the action of $G$ on $\Omega$, namely the connected components of $(\Omega,\Gamma)$. They coincide with the sets ${^U \omega}$.

Moreover: ${^U \omega}$ is the smallest possible block containing both $\omega$ and ${^y \omega}$.
\end{remark}


\section{Orbital graphs and representation theory}

\begin{definition}
Let $\Omega$ be a finite $G$-set. Then define $\IK^\Omega$ is defined  as the $\mathbb{K}$-vector space with basis $\Omega$. This vector space is naturally a $\IK G$-module by extending the action of $G$ on the basis elements linearly to the whole space.

We also define $Perm(\Omega) \leq GL(\IK^\Omega)$ to be the set of all permutation matrices acting on $\IK^\Omega$.

The representation $G\to Perm(n)$ will be denoted by $\rho_\Omega$.
\end{definition}

\begin{remark}
We mainly introduce $Perm(\Omega)$ to have a clean notational distinction between a permutation and its associated permutation matrix.
\end{remark}

\begin{theorem}
$\End_{\IK G}(\IK^\Omega)$ has a natural $\IK$-basis $\set{X_\Gamma | \Gamma\subseteq\Omega^2\;\text{orbital}}$ defined as
\[(X_\gamma)_{ij} := \begin{cases} 1 & \text{if}\;(i,j)\in\Gamma \\ 0 & \text{otherwise}\end{cases}\]

The structure constants w.r.t. this basis, i.e. the numbers $d_{ij}^k$ such that
\[X_{\Gamma_i} \cdot X_{\Gamma_j} = \sum_{k} d_{ij}^k X_{\Gamma_k},\]
are given by $d_{ij}^k := \abs{\Set{\beta\in\Omega | (\alpha,\beta)\in\Gamma_i \wedge (\beta,\gamma) \in \Gamma_j}}$ where $(\alpha,\gamma)$ is any element of $\Gamma_k$.
\end{theorem}
\begin{remark}
In other words: $X_\Gamma$ is the adjacency matrix of the orbital graph $(\Omega,\Gamma)$.

Note that the right hand side in the definition of $d_{ij}^k$ really is independent of the choice of the element $(\alpha,\gamma)\in\Gamma_k$, because $G$ acts transitively on $\Gamma_k$.

Also note that multiplication is connected to composition via
\[X_{\Gamma_i} \cdot X_{\Gamma_j} \in \LinHull_\IK\set{X_\Gamma | \Gamma\subseteq \Gamma_i\circ\Gamma_j}\]
\end{remark}
\begin{proof}
Writing out the defining condition
\[X\in\End_{\IK G}(\IK^\Omega) \iff \forall g\in G: gXg^{-1} = X\]
in components shows that every $\IK G$-linear endomorphism is a linear combination of the $X_\Gamma$. They are obviously linear independent and therefore a basis.

The structure constants similarly follow by writing out the definition of matrix multiplication in this case.
\end{proof}

\begin{remark}
$\End_{\IC G}(\IK^\Omega)$ is by definition the centraliser algebra of the subalgebra $\LinHull_\IK(\rho_\Omega(G)) \subseteq \IK^{\Omega\times\Omega}$.

If $\IK$ has characteristic zero, $\IK G$ is semisimple and therefore all modules have the double centraliser property so that $C(\End_{\IK G}(\IK^\Omega)) = \LinHull_\IK(\rho_\Omega(G))$.
\end{remark}

\begin{definition}
The \udot{2-closure} of a permutation group $G\leq Sym(\Omega)$ is defined as the largest subgroup $\widehat{G}\subseteq Sym(\Omega)$ that has the same orbits as $G$ on $\Omega^2$, i.e.
\[\widehat{G} := \Set{\pi\in Sym(\Omega) | \forall \Gamma\in \Omega^2/G: \pi(\Gamma) = \Gamma}\]
$G$ is called \udot{2-closed} iff $G=\widehat{G}$ holds.
\end{definition}

\begin{lemma}[2-closure in terms of endomorphism algebras]\label{two_closure:in_terms_of_endomorphisms}
The 2-closure of $G$ is the largest subgroup $H\leq Sym(\Omega)$ that still satisfies $End_{\IK G}(\IK^\Omega) = End_{\IK H}(\IK^\Omega)$, i.e.
\[\widehat{G} = Perm(\Omega) \cap C(\End_{\IK G}(\IK^\Omega)).\]
In particular, $G$ is 2-closed if no permutation matrix outside $G$ commutes with all endomorphisms of $\IK^\Omega$.
\end{lemma}
\begin{proof}
Let $\widehat{G}$ be the 2-closure of $G$. By definition $\pi\in\widehat{G}$ if and only if $\pi X_\Gamma \pi^{-1} = X_\Gamma$ for all $\Gamma\in\Omega^2/G$. In other words $\pi$ is in the 2-closure iff it is a permutation and an element of the centraliser of the endomorphism ring of the $\IK G$-module $\IK^\Omega$.
\end{proof}

\begin{theorem}[2-closure in terms of invariant subspaces]\label{two_closure:in_terms_of_subspaces}
$\widehat{G} = \Set{\gamma\in Perm(\Omega) | \forall U\leq\IC^\Omega: U\;G\text{-invariant} \implies U\;\gamma\text{-invariant}}$
\end{theorem}
\begin{proof}
We consider the standard scalar product on $\IC^\Omega$ defined by declaring $\Omega$ to be an orthonormal basis.

Then all permutation matrices are unitary. In particular, $\LinHull_\IC(\rho_\Omega(G))$ is closed under taking adjoints and its centraliser $\End_{\IC G}(\IC^\Omega)$ is also closed under taking adjoints. Both are therefore $C^\ast$-algebras. In particular, both are isomorphic to a direct product of matrix rings. It is a consequence of the spectral theorem that $\prod_i \IC^{n_i\times n_i}$ is spanned by all the self-adjoint idempotents it contains.

\medbreak
Self-adjoint idempotent matrices correspond bijectively to subspaces by identifying $U$ with the orthogonal projection $p_U$ onto $U$. A subspace $U$ is $g$-invariant if $g$ centralises $p_U$.

Therefore
\[\End_{\IC G}(\IC^\Omega) = \LinHull_\IC\Set{p_U | U\leq\IC^G\;G\text{-invariant}}\]
and
\[\widehat{G} = Perm(\Omega) \cap C(\End_{\IC G}(\IC^\Omega)) = Perm(\Omega) \cap \bigcap_{\substack{U\leq\IC^\Omega \\ G\text{-invariant}}} C(p_U)\]
which proves the claim.
\end{proof}

\begin{definition}
A permutation group $G\leq Sym(\Omega)$ is
\begin{itemize}
\item \udot{reconstructible from $\mathcal{X}\subseteq\End_{\IK G}(\IK^\Omega)$} if $G = Perm(\Omega) \cap \bigcap_{X\in\mathcal{X}} C(X)$,
\item \udot{orbital-graph-reconstructible} if $G$ is reconstructible from $\Set{X_\Gamma | \Gamma\in\Omega^2/G}$,
\item \udot{strongly orbital-graph-reconstructible from  $\Gamma\in\Omega^2/G$} iff it is reconstructible from $X_\Gamma$ alone,
\item \udot{absolutely orbital-graph-reconstructible} iff it is strongly orbital-graph-reconstrucible from any non-diagonal $\Gamma\in\Omega^2$.
\item \udot{subspace-reconstructible} from a set $\mathcal{U}$ of $G$-invariant subspaces of $\IK^\Omega$ if $G$ is reconstructible from $\Set{p_U | U\in\mathcal{U}}$.
\item \udot{strongly subspace-reconstructible} from $U\leq\IK^\Omega$ if $G$ is reconstructible from $p_U$ alone,
\item \udot{absolutely subspace-reconstructible over $\IK$} if $G$ is strongly subspace-reconstructible from any $p_U$ where $U\leq\IK^G$ is a minimal, non-zero $G$-invariant subspace which is not $\LinHull_\IK\set{(1,1,\ldots,1)}$.
\end{itemize}
\end{definition}

\begin{corollary}
$G\leq Sym(\Omega)$ is 2-closed iff it is orbital-graph reconstructible iff it is subspace-reconstructible over $\IC$.
\end{corollary}
\begin{proof}
The first equivalence follows from the fact that $X_\Gamma$ is a basis of $\End_{\IC G}(\IC^\Omega)$. The second follows from theorem \ref{two_closure:in_terms_of_subspaces}.
\end{proof}

\begin{example}
A regular permutation group is always 2-closed.

This is because a regular $G$-set is isomorphic to $G$ itself endowed with left multiplication. The orbitals of this action are given by $\Gamma_h:=\set{(x,y)\in G^2 | x^{-1} y=h}$ for $h\in G$ and one can readily verify that the only permutations fixing all the orbitals are the left multiplication maps themselves.
\end{example}

\begin{lemma}
Let $\Gamma$ be an orbital. Then $G$ is strongly orbital-graph-reconstructible from $\Gamma$ iff if strongly orbital-graph-reconstructible from $\Gamma^{op}$ iff it is subspace-reconstrucible from
\[\Set{\operatorname{Eig}_\lambda(\Re(X_\Gamma)),\operatorname{Eig}_\lambda(\Im(X_\Gamma)) | \lambda\in\IR}.\]
\end{lemma}
\begin{proof}
Permutation matrices are unitary. Therefore $g\in Perm(n)$ centralises $X$ iff it centralises $X^\ast$. The adjoint of $X_\Gamma$ is exactly $X_{\Gamma^{op}}$ as we have previously observed.

$\Re(X) = \frac{1}{2}(X+X^\ast)$ and $\Im(X)=\frac{1}{2i}(X-X^\ast)$ are self-adjoint matrices with $X=\Re(X)+i\Im(X)$ and for a self-adjoint matrices $Y$ the spectral theorem shows
\[Y = \sum_{\lambda\in\IR} \lambda e_\lambda\]
where $e_\lambda=p_{\operatorname{Eig}_\lambda(Y)}$ is the orthogonal projection onto the $\lambda$-eigenspace. Moreover $e_\lambda$ is a polynomial of $Y$ by Lagrange-interpolation.

Therefore if $g\in GL(\IK^\Omega)$ commutes with $Y$ it must commute with all $e_\lambda$ and vice versa. Thus
\[C(X_\Gamma) = C(X_\Gamma,X_\Gamma^\ast) = C(\Re(X_\Gamma),\Im(X_\Gamma))=\bigcap_{\lambda} C(p_{\operatorname{Eig}_\lambda(\Re(X_\Gamma))}) \cap C(p_{\operatorname{Eig}_\lambda(\Im(X_\Gamma))})\]
which proves the lemma.
\end{proof}

\begin{remark}
The concept of subspace reconstructibility also makes sense if we replace $Perm(\Omega)$ by some other finite subgroup, for example the subgroup of monomial matrices.
\end{remark}

\end{document}